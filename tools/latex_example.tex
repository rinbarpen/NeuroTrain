% LaTeX文档示例 - 展示如何使用data_to_latex.py生成的代码
\documentclass{article}
\usepackage[utf8]{inputenc}
\usepackage{longtable}  % 用于长表格
\usepackage{graphicx}
\usepackage{booktabs}   % 用于更美观的表格

\title{Model Performance Comparison}
\author{NeuroTrain}
\date{\today}

\begin{document}

\maketitle

\section{Introduction}
This document demonstrates how to use LaTeX code generated by the \texttt{data\_to\_latex.py} tool.

\section{Model Comparison Table}

Below is a comparison table of different neural network models generated from CSV data:

% 生成命令:
% python tools/data_to_latex.py -i tools/example_data.csv -t table 
%   --caption "Model Performance Comparison" --label "tab:models"

\begin{table}[htbp]
\centering
\caption{Model Performance Comparison}
\label{tab:models}
\begin{tabular}{lllll}
\hline
model & accuracy & f1\_score & params & task \\
\hline
ResNet50 & 0.9523 & 0.9412 & 25.6M & classification \\
VGG16 & 0.9234 & 0.9145 & 138.4M & classification \\
EfficientNet & 0.9678 & 0.9589 & 5.3M & classification \\
MobileNet & 0.9012 & 0.8934 & 4.2M & classification \\
U-Net & 0.8834 & 0.8756 & 31.0M & segmentation \\
DeepLabV3 & 0.9145 & 0.9023 & 58.7M & segmentation \\
\hline
\end{tabular}
\end{table}

Table \ref{tab:models} shows the performance metrics of various models. 
As we can see, EfficientNet achieves the highest accuracy with the smallest 
parameter count, making it the most efficient model.

\section{Model List}

Here is an itemized list of models:

% 生成命令:
% python tools/data_to_latex.py -i tools/example_data.csv -t itemize 
%   --template "{model}: Accuracy={accuracy}, F1={f1_score}"

\begin{itemize}
  \item ResNet50: Accuracy=0.9523, F1=0.9412
  \item VGG16: Accuracy=0.9234, F1=0.9145
  \item EfficientNet: Accuracy=0.9678, F1=0.9589
  \item MobileNet: Accuracy=0.9012, F1=0.8934
  \item U-Net: Accuracy=0.8834, F1=0.8756
  \item DeepLabV3: Accuracy=0.9145, F1=0.9023
\end{itemize}

\section{Model Descriptions}

% 生成命令:
% python tools/data_to_latex.py -i tools/example_data.csv -t description 
%   --key-column model

\begin{description}
  \item[ResNet50] accuracy: 0.9523, f1\_score: 0.9412, params: 25.6M, task: classification
  \item[VGG16] accuracy: 0.9234, f1\_score: 0.9145, params: 138.4M, task: classification
  \item[EfficientNet] accuracy: 0.9678, f1\_score: 0.9589, params: 5.3M, task: classification
  \item[MobileNet] accuracy: 0.9012, f1\_score: 0.8934, params: 4.2M, task: classification
  \item[U-Net] accuracy: 0.8834, f1\_score: 0.8756, params: 31.0M, task: segmentation
  \item[DeepLabV3] accuracy: 0.9145, f1\_score: 0.9023, params: 58.7M, task: segmentation
\end{description}

\section{Conclusion}

The \texttt{data\_to\_latex.py} tool makes it easy to convert data files 
into properly formatted LaTeX code. This is especially useful for:

\begin{enumerate}
  \item Writing research papers
  \item Creating technical reports
  \item Generating documentation
  \item Presenting experimental results
\end{enumerate}

\subsection{Tips for Using the Tool}

\begin{itemize}
  \item Always specify a caption and label for tables to make them easier to reference
  \item Use \texttt{longtable} for tables that span multiple pages
  \item Customize templates for itemize/enumerate lists to match your needs
  \item Use the \texttt{--show-info} option to preview your data before conversion
  \item Select only necessary columns with \texttt{-c} to keep tables concise
\end{itemize}

\end{document}

